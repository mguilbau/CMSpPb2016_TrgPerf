\section{QGP fluid in small systems}

\begin{figure}[thb]
  \begin{center}
    \includegraphics[width=0.48\textwidth]{figures/corr2D_pPb_N110_pt1-3_20121016.pdf}
    \includegraphics[width=0.48\textwidth]{figures/v2m_pPb.pdf}
    \caption{ Left: The 2D two-particle correlation functions in high-multiplicity 
    pPb collisions at \rootsNN\ = 5.02 TeV measured by the CMS experiment.
    Right: The elliptic anisotropy, $v_2$, as a function of $N_{\rm trk}$
    obtained from two-, four-, six- and eight-particle cumulants, and the LYZ method, 
    averaged over $0.3<\pt<3.0$~GeV/c, in pPb collisions at \rootsNN\ = 5.02 TeV.
  %~\cite{Khachatryan:2015waa}.
    }
    \label{fig:ridge_pPb}
  \end{center}
\end{figure} 

Observation of a long-range, near-side structure (often called the ``Ridge'') in two-particle
\deta\ -- \dphi\ correlation function of high-multiplicity pp~\cite{Khachatryan:2010gv} 
and pPb~\cite{CMS:2012qk} collisions at CMS (Fig.~\ref{fig:ridge_pPb} left)
opened up new opportunities for studying novel dynamics of particle production 
in small but high-density Quantum Chromodynamic (QCD) systems. Similar ridge-like structure is
first observed in relativistic nucleus-nucleus (AA) collisions, and is widely accepted 
to be originated from the hydrodynamic collective flow of a strongly interacting and 
expanding medium. 

In small colliding systems like pp and pPb, there has not been a consensus 
in the community on the origin of the ridge-like correlation structure. 
While the formation of a hot and dense fluid-like medium provides a natural explanation, 
it was not expected because the overlapping region is significantly smaller than 
that in an AA collision. Various alternative theoretical interpretations have been 
proposed, such as models of initial-state gluon correlations without any final-state interactions.
A key observation made by CMS in 2013 pPb run at \rootsNN\ = 5.02 TeV is that
the elliptic azimuthal anisotropy harnomics, $v_2$, extracted using four-, six-,
eight- and all particle azimuthal correlations, are found to be nearly identical
within about 10\%~\cite{Khachatryan:2015waa}, lending strong support to the highly 
collective nature of correlations in these systems. 

While much progress has been made in the LHC run 1, many questions still remain unanswered:
(1) What is the underlying mechanism driving the observed collective behavior of particles? 
Strong final-state rescatterings in an initially eccentric system (like AA collisions) or 
initial momentum-space correlations due to initial-state gluon interactions? (2) If a hot and opaque 
medium is indeed formed in a pPb collision, how does it influence the behavior of hard probes
such as jets and quarkonia? Higher energy and luminosity pPb collisions at the LHC run 2 
will enable a wide range of high precision measurements, providing us unique opportunities 
to address these important questions. 