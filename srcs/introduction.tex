\section*{Introduction}

Following a successful, first-time ever run of pPb collisions 
at \rootsNN\ = 5.02 TeV at the LHC in 2013 (an integrated luminosity 
of 35 nb$^{-1}$), a new pPb run is going to take place at the 
LHC in 2016, with highest possible collision energy of 8.16 TeV.

There are two major physics goals of the pPb program at CMS:

\begin{itemize}
\item Explore possible formation of quark-gluon plasma in high-multiplicity, small colliding systems. 
\item Study the modification of parton distribution functions in nuclei (nPDF) over 
an unexplored regime in Bjorken-$x$ and $Q^{2}$. 
\end{itemize}

In order to fulfill the two major physics goals, 
CMS would like to request pPb collisions at highest achievable 
energy of \rootsNN\ = 8.16 TeV with the highest possible integrated 
luminosity to be delivered. A number of physics cases are presented 
in this note to justify this request. For illustration purpose, 
an integrated luminosity of 100 nb$^{-1}$ (a desired and achievable value) 
is assumed in making projections of various measurements,
and evaluating statistical and systematic uncertainties.